\let\negmedspace\undefined
\let\negthickspace\undefined
\documentclass[journal,12pt,twocolumn]{IEEEtran}
\usepackage{cite}
\usepackage{amsmath,amssymb,amsfonts,amsthm}
\usepackage{algorithmic}
\usepackage{graphicx}
\usepackage{textcomp}
\usepackage{xcolor}
\usepackage{txfonts}
\usepackage{listings}
\usepackage{enumitem}
\usepackage{mathtools}
\usepackage{gensymb}
\usepackage{comment}
\usepackage[breaklinks=true]{hyperref}
\usepackage{tkz-euclide} 
\usepackage{listings}
\usepackage{gvv}  
\usepackage{tikz}
\usepackage{circuitikz} 
\usepackage{caption}
\def\inputGnumericTable{}              
\usepackage[latin1]{inputenc}          
\usepackage{color}                    
\usepackage{array}                     
\usepackage{longtable}                 
\usepackage{calc}                     \usepackage{multirow}                  
\usepackage{hhline}                    
\usepackage{ifthen}                    
\usepackage{lscape}
\usepackage{amsmath}
\newtheorem{theorem}{Theorem}[section]
\newtheorem{problem}{Problem}
\newtheorem{proposition}{Proposition}[section]
\newtheorem{lemma}{Lemma}[section]
\newtheorem{corollary}[theorem]{Corollary}
\newtheorem{example}{Example}[section]
\newtheorem{definition}[problem]{Definition}
\newcommand{\BEQA}{\begin{eqnarray}}
\newcommand{\EEQA}{\end{eqnarray}}
\newcommand{\define}{\stackrel{\triangle}{=}}
\theoremstyle{remark}
\newtheorem{rem}{Remark}

%\bibliographystyle{ieeetr}
\begin{document}
%

\bibliographystyle{IEEEtran}




\title{
%	\logo{
Analog 12.7.4

\large{EE1205 : Signals and Systems}

Indian Institute of Technology Hyderabad
%	}
}
\author{Chirag Garg

(EE23BTECH11206)
}	





\maketitle

\newpage



\bigskip

\renewcommand{\thefigure}{\theenumi}
\renewcommand{\thetable}{\theenumi}


%\section{Question 12.7.4}


\textbf{Question:} A 60 $\mu$ F capacitor is connected to a 110 V, 60 Hz ac supply. Determine
the rms value of the current in the circuit.


%\section{Solution} 
\textbf{Solution: }
\begin{figure}[!h]
 \centering
    \begin{circuitikz}
   \draw (0,3)
    to[C, C=$60\mu F$] (2,3)
          to[sV, v=$110\,V\text{, 50Hz}$] (6,3);
    
    % Draw the horizontal lines to complete the circuit
    \draw (0,0) -- (6,0);
    \draw (0,0) -- (0,3);    
    \draw (6,0) -- (6,3);
    \end{circuitikz}
    \caption*{Fig. 1}
    \label{fig:enter-label}
\end{figure}
\begin{table}[htbp]
\centering
\resizebox{\columnwidth}{!}{
\begin{tabular}{|c|c|c|}
    \hline
     \textbf{Symbol} & \textbf{Value} &
     \textbf{Description}\\
   
    \hline 
     $C$ &  $60\, \mu F$ & Capacitance\\
     
    \hline
     $V_0$ & $110 \sqrt{2}V$ & Peak Voltage  \\
      \hline
       $I_0$ & $V_0 \times \omega C$ & Peak Current  \\
      \hline
      
      $f$ & $60 Hz$ & Frequency \\
   
    \hline
     $\omega$ & $ 2\pi f$ & Angular Frequency\\
      \hline
      
     $H(s)$ & $\dfrac{V(s)}{I(s)}$ & Transfer Function \\ 
        \hline

\end{tabular}
}

\caption*{Table 1 : Given Parameters}
\label{tab:my_label}
\end{table}

Substituting values: \\
\begin {align}
X_C = \dfrac{1}{2\pi \times 60 \times 60 \times 10^{-6}} \Omega
\end{align}
\begin {align}
s=j\times2\times\pi\times60 \: T^{-1}
\end{align}
The $I_{\text{rms}}$ value is defined as :-
\begin{align}
I_{\text{rms}}^2 &= {\dfrac{1}{T} \int_{0}^{T} [I(t)]^2 \, dt} \\
&= {f \int_{0}^{\frac{1}{f}} I_{\text{0}}^2 \cdot \sin^2(2\pi ft + \phi) \, dt} \\
&= \dfrac{1}{2} I_{0}^2 \left(1 - \frac{1}{f}\left[\dfrac{\sin(4\pi ft + 2\phi)}{4\pi f}\right]_{0}^{\frac{1}{f}}\right) \\
&= \dfrac{1}{2} I_{0}^2 \left(1 - \dfrac{1}{f} \cdot \dfrac{\sin\left(4\pi + 2\phi\right) - \sin(0 + 2\phi)}{4\pi f}\right) \\
&= \dfrac{I_{0}}{\sqrt{2}} 
\end{align}
\begin{figure}[htbp]
 \centering
    \begin{circuitikz}
   \draw (0,3)
    to[C, C=${\raisebox{0.8ex}{$\dfrac{1}{sC}$}}$] (2,3)
          to[sV, v=$V(s)$] (6,3);
    
    % Draw the horizontal lines to complete the circuit
    \draw (0,0) -- (6,0);
    \draw (0,0) -- (0,3);    
    \draw (6,0) -- (6,3);
    \end{circuitikz}
    \caption*{Fig. 2 : Equivalent s domain circuit}
    \label{fig:enter-label}
\end{figure} \\
We know that,
\begin{align}
 H(s)&=R + sL + \dfrac{1}{sC}\\
&=0 + 0 +\dfrac{1}{j \omega C}\\
&=\dfrac{1}{j \omega C}\\
\end{align}
\begin{align}
 |H(j \omega)| &= \sqrt{ \dfrac{1}{\omega^2 C^2}}\\
 &= \dfrac{1}{{ (2\pi) \times(60)\times(60) \times (10^{-6})}}\\
&\approx 44.21\\
I(s) &= \dfrac{V(s)}{H(s)} \\
&=\dfrac{110}{44.21} \\
&\approx 2.49  A
\end{align}

\end{document}
